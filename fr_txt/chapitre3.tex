\chapter{La conférence}
Une conférence de physique est sans hésiter un événement que chacun
devrait vivre au moins une fois dans sa vie. La faune y est d'une
diversité biodômesque. On y retrouve les jeunes étudiants gradués, pas
certains de comprendre ce qu'ils font là. Ils ne se sentent pas tout à
fait à leur place, craignent qu'on débusque leur ignorance, plus
souvent bien réelle qu'imaginée, mais sont tout excités par l'occasion
de voyager gratuitement. Ils écoutent très peu les autres
présentations, et peinent souvent à rester éveillés durant celles-ci
en raison d'une faible tolérance au décalage horaire et d'un sevrage
alcoolique leur demandant visiblement toute leur énergie.

Il y a aussi les étudiants gradués plus avancés et les
post-docs. Blasés des voyages, ils sont là pour présenter les travaux
des stagiaires de premier cycle ou de leur superviseur qui “n'a pas pu
se libérer” ou encore pour détailler pour la centième fois l'analyse
sur laquelle ils ont travaillé durant les cinq dernières
années. Souvent les experts mondiaux dans leur domaine, ils préparent
leur présentation le matin même pendant celles des autres, et
communiquent ensuite le résultat avec une rigueur et une précision
irréprochable. Ils échouent cependant à cacher à quel point le sujet
les ennuie, à évacuer l'amertume de leur regard, ainsi que le regret
qui s'y trouve. Le regret d'avoir fait ces choix de vie qui ont fait
d'eux des grands enfants à l'intelligence relationnelle aussi nulle
que leurs talents au lit, vivant à 31 ans d'un revenu qui, si l'on
compte les heures travaillés, se compare au salaire minimum malgré les
diplômes dont ils croyaient qu'ils seraient fiers. La plupart sont au
courant que le Saint-Graal d'un poste permanent à une université
respectable est tout simplement hors de portée, mais se résignent à
“finir ce qu'ils ont commencé” sans trop être certain de ce que ça
leur apporte.

Une cause de cette situation, selon plusieurs, est l'acharnement des
vieux profs. À tout avoir sacrifié pour cette carrière, ils croient
qu'il y ont droit jusqu'à leur mort. Au plafond de l'échelle
salariale, ces scientifiques démontrent une rare résistance au
changement et à la nouveauté. Il n'est pas rare que leur communication
soit retardée car on ne trouve pas de lampe appropriée pour le
rétroprojecteur, ou encore lorsqu'il n'y a pas de brosse à effacer
pour le tableau. Ils écoutent toutes les présentations, posent des
questions avec de longs préambules, sont au sommet de leur forme au
petit matin mais au tournent ralenti à l'heure de la sieste de
l'après-midi. Il regardent aussi avec envie tous les autres “jeunes”
et se remémorent le temps où ils venaient à ces conférences pour se
saoûler, se sauver de leur femme et sauter les trop rares filles
travaillant dans le même domaine qu'eux.

C'est dans cet enclos que je fais mon entrée. Bien qu'au milieu d'une
présentation, personne ne remarque mon arrivée tardive, l'essentiel du
public a les yeux rivés sur un ordinateur portable. Cet essentiel du
public est composé de gens identiques à moi. Entre mi-trentaine et la
mi-quatrantaine, d'une apparence soignée sans être nécessairement au
goût du jour (il faut l'admettre!). Même si nous sommes bien au fait
de la vacuité de tous nos efforts, nous demeurons d'ambitieux
chercheurs travaillant dur pour tenter de gagner l'estime de la
“communauté” en espérant ne pas réaliser que cette estime ne vendra
jamais compléter le peu de respect que nous avons pour nous
même. Certains, comme moi, ont une femme ou même une famille, mais
celles-ci n'ont maheureusement pas la place qu'elles méritent dans la
liste de nos préoccupations.  La carrière prend le dessus, et ceci
s'exprime autant par le faible nombre de jours que nous passons dans
notre propre ville que par le faible nombre d'heures de qualité passé
en leur compagnie. La maison nous sert de dortoir entre deux journées
au bureau. Pensez Up in the Air avec George Clooney. D'une façon assez
surprenante, nous nous trouvons la plupart du temps satisfaits de
cette situation. L'accès fréquent à ces grandes messes, où tous se rencontrent
sous de sérieux prétextes dans des villes exotiques pour festoyer et
diversifier leur vie sexuelle n'y est pas étranger.

Je m'asseyai donc dans l'avant dernière rangée, jetant des regards
furetifs à l'audience pour voir si une tête familière ou un regard
inquisiteur m'appapraîtra et déterminer avec qui j'ai bien pu terminer
la soirée de la veille. Par chance j'avais terminé mes “diapos” dans
l'avion et je connaissais bien le sujet (l'essentiel de la
présentation serait recyclée du travail de mes étudiants, avec leur
consentement, bien sûr!). Je déballai mon contenu efficacement,
répondis aux quelques questions, balayé la pièce une dernière fois
avant de déscendre de scène pour me convaincre que non, des 100
personnes présentes, pas plus de 15 étaient des femmes, aucune ne me
regardait avec ce particulier regard du lendemain, aucune non plus
n'avait le potentiel de partager mon lit. S'il était très peu probable
que je trompe un jour Marie, il est tout simplement impossible que je
couche avec une femme que je ne désire pas. Même sous les effets de
l'alcool, aussi puissants soient-ils, je sais que j'ai un goût très
pointu pour les femmes. Aucune ne remplit les critères.

Aussitôt redescendu de scène, je me dirige tranquilement vers la
sortie sans que personne ne le remarque. Le prochain présentateur est
en train d'ouvrir son document, personne n'a détourné les yeux de leur
portable. Je monte à ma chambre dans l'espoir de rattraper quelques
heures de sommeil, et comprendre un peu mieux le pétrin dans lequel je
suis plongé.